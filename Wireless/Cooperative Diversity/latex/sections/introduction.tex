\IEEEPARstart{W}{ireless} ad-hoc networks have recently become a topic of research as they can be deployed quickly in locations lacking the infrastructure required for typical wireless networks.

Like any wireless network, ad-hoc networks would benefit greatly from the diversity advancements made in the last couple decades.

Compared to the transmitters and receivers in traditional wireless networks, the nodes in an ad-hoc network may be much smaller and possibly mobile.
This discourages the use of many techniques for increasing capacity, most notably the use of multiple antennas at the transmitter for obtaining a diversity gain.
A diversity strategy is proposed in \cite{4686273} that creates a virtual antenna array by utilizing inactive neighboring nodes in transmitting with an orthogonal space-time block code.

The key measurement in this paper is throughput, as high throughput in a wireless network is the defining metric of a good connection.  Therefore, we will attempt to validate the results by means of comparing the throughput of a system with that of an identical system implementing an adaptive cooperative diversity scheme. Throughput in this paper is defined as the ratio of the number of packets successfully transmitted to the total number of packets transmitted, under the assumption that the overhead of obtaining diversity nodes is insignificant compared to the transmission time of a packet.